A {\ttfamily parser} class defines a running environement for transcription by parsing for a given {\ttfamily input}. It assembles elements from the {\ttfamily table} directory for the construction of a table used to compute a tree from a {\ttfamily grammar} and some {\ttfamily input}.

Each {\ttfamily parser} class may contain a {\ttfamily demo} fonction to be called in a target.

The following {\ttfamily parsers} have been implemented\+:


\begin{DoxyItemize}
\item {\ttfamily Inputless1best} \+: compute the 1-\/best tree of a {\ttfamily \mbox{\hyperlink{classWTA}{W\+TA}}}. no input segment.
\item {\ttfamily Inputlesskbest} \+: compute the k best trees of a {\ttfamily \mbox{\hyperlink{classWTA}{W\+TA}}}. no input segment.
\item {\ttfamily 1bar1best\+S\+IP} \+: computing the 1-\/best tree in a given {\ttfamily \mbox{\hyperlink{classWTA}{W\+TA}}} language for the transcription of a given input segment. If the \mbox{\hyperlink{classWTA}{W\+TA}} trees represent 1 bar, this scenario is transcription of the whole segment as a single bar.
\item {\ttfamily 1barkbest\+S\+K\+IP} \+: computing the k best trees in a given {\ttfamily \mbox{\hyperlink{classWTA}{W\+TA}}} language for the transcription of a given input segment.
\item {\ttfamily Multibar1best\+S\+I\+P\+BU} \+: 1-\/best parsing with {\ttfamily S\+IP} pointers. Process input as multiple bars, where a sequence of bars is represented by a binary tree (meta-\/run), constructed in a bottom-\/up fashion\+:
\end{DoxyItemize}


\begin{DoxyCode}{0}
\DoxyCodeLine{                            [bars 1-n]}
\DoxyCodeLine{                             /     \(\backslash\)}
\DoxyCodeLine{                 [bars 1-(n-1)]   bar m}
\DoxyCodeLine{               ...}
\DoxyCodeLine{      [bars 1-2]   bar3}
\DoxyCodeLine{       /      \(\backslash\)}
\DoxyCodeLine{  [bars 1]   bar2}
\DoxyCodeLine{  /    \(\backslash\)}
\DoxyCodeLine{[ ]   bar1}
\end{DoxyCode}


Every node corresponds to a S\+IP pointer. The nodes {\ttfamily p1}, {\ttfamily p2} immediately below a node {\ttfamily p} represent a binary run {\ttfamily (p1, p2)} in the table entry for {\ttfamily p}. The pointers in {\ttfamily \mbox{[} \mbox{]}} correspond to several bars (meta pointers), they do have a non-\/\+W\+TA state. Every other pointer correspond to a single bar, with the initial \mbox{\hyperlink{classWTA}{W\+TA}} state (and contain a best run for that bar).

It is assumed that the number of bars is known and the bar length is fixed (tempo does not vary from bar to bar).

This parser can be used for online (bar by bar) transcription.


\begin{DoxyItemize}
\item {\ttfamily Multibar1best\+S\+I\+Pflat} \+: same as above but the sequence of bars is represented by a tuple (flat tree), constructed from left to right.
\end{DoxyItemize}


\begin{DoxyCode}{0}
\DoxyCodeLine{ [bars 1-n]}
\DoxyCodeLine{  /      \(\backslash\)}
\DoxyCodeLine{bar1 ... bar m}
\end{DoxyCode}


Every node correspond to a S\+IP pointer. The top note correspond to the whole segment (meta pointer for all bars) Every node below correspond to a single bar, with the initial \mbox{\hyperlink{classWTA}{W\+TA}} state (and contain a best run for that bar). This parser cannot be used for online transcription.

This parser is very inefficient with constraint solving (in \mbox{\hyperlink{classTable}{Table}}) for pre, post values -\/ need to store an exponential number of partial runs. The BU version is more efficient -\/ with more compact representation of partial runs. 